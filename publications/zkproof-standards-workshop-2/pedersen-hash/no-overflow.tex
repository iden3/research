As we described in section \ref{sec-computation}, 
%We have seen 
we use a windowed scalar multiplication algorithm with signed digits. Each 4-bit message chunk corresponds to a window called \textsc{selector} and each chunk is encoded as an integer from the set $\{-8..8\}\backslash \{0\}$. % and we allow to have up to $50$ windows. 
This allows a more efficient lookup of the window entry for each chunk than if the set $\{1..16\}$ had been used, because a point can be conditionally negated using only a single constraint \cite{sapling}.\\
% Sapling, the new network upgrade of Zcash, uses 3-bit windows but we propose to use 4-bit windows, as it requires less constraints per bit.\\

As there are up to 50 segments per each generator $P_i$, the largest multiple of the generator $P_i$ is $n\cdot P_i$ with 
$$n = 2^0 \times8 + 2^5 \times 8 + \left(2^5\right)^2 \times8 \dots + 	2^{245}\times 8 .$$
To avoid overflow, this number should be smaller than $(r-1)/2$. Indeed,
%We have to make sure that this number is smaller than $(r-1)/2$, where $r$ is the order of the large prime subgroup of the curve.  Indeed,
\begin{align*}
	\quad\; n 
	& = 8 \times \sum_{ k = 0}^{49} 2^{5k}
	= 8 \times \frac{2^{250}-1}{2^5-1}\\
	& = 466903585634339497675689455680193176827701551071131306610716064548036813064%\\
\end{align*}
\vspace{-0.2cm}
and 
\begin{align*}
	\frac{r-1}{2} &= 1368015179489954701390400359078579693038406986079283629600107830474223686520 \\
	& > n.\\ \vspace{0.4cm}
\end{align*}