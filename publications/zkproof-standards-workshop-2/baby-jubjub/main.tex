% % !TEX root =/Users/martabellesmunoz/Dropbox/Documents/Berkley-Standards/Structure.tex

\documentclass{article}
\usepackage[colorlinks=true,
			urlcolor=black,
			linkcolor=black,
			citecolor=black
			]{hyperref} 

\usepackage[english]{babel}
\usepackage[utf8]{inputenc}
\usepackage{amsmath, amsthm, amssymb}
\usepackage{enumerate, enumitem}
\usepackage{graphicx}
\usepackage{color, xcolor}
\usepackage{setspace}
\usepackage{hyperref}
\usepackage{authblk}
\usepackage{tikz}
\usetikzlibrary{arrows}
\usetikzlibrary{positioning}
\usepackage{mathtools}
%floor vs. ceil
\DeclarePairedDelimiter{\floor}{\lfloor}{\rfloor} 
\DeclarePairedDelimiter{\ceil}{\lceil}{\rceil}  % If called \ceil*{x} it will add left/right.

%\usepackage{algorithmicx}
\usepackage{algorithm}
\usepackage[noend]{algpseudocode}
\makeatletter
\def\BState{\State\hskip-\ALG@thistlm}
\makeatother
%\usepackage{listings}
%\lstdefinelanguage{Sage}[]{Python}
%{morekeywords={False,sage,True},sensitive=true}
%\lstset{
%  frame=none,
%  showtabs=False,
%  showspaces=False,
%  showstringspaces=False,
%  commentstyle={\ttfamily\color{dgreencolor}},
%  keywordstyle={\ttfamily\color{dbluecolor}\bfseries},
%  stringstyle={\ttfamily\color{dgraycolor}\bfseries},
%  language=Sage,
%  basicstyle={\fontsize{10pt}{10pt}\ttfamily},
%  aboveskip=0.3em,
%  belowskip=0.1em,
%  numbers=left,
%  numberstyle=\footnotesize
%}


\textwidth 16 cm
\textheight 22 cm
\topmargin -1 cm
\oddsidemargin -0 cm

\addtolength{\skip\footins}{1pc plus 5pt} % Foot note space
\setlength\parindent{0pt} % No indent

\newcommand{\N}{\ensuremath{\mathbb{N}}}
\newcommand{\Np}{\ensuremath{\mathbb{N}^{+}}}
\newcommand{\Z}{\ensuremath{\mathbb{Z}}}
\newcommand{\Q}{\ensuremath{\mathbb{Q}}}
\newcommand{\R}{\ensuremath{\mathbb{R}}}
\newcommand{\C}{\ensuremath{\mathbb{C}}}
\newcommand{\Fp}{\ensuremath{\mathbb{F}_p}}
\newcommand{\Fr}{\ensuremath{\mathbb{F}_r}}
\newcommand{\G}{\ensuremath{\mathbb{G}}}
\newcommand{\point}[1]{P_{#1} = (x_{#1}, y_{#1})}
\newcommand{\llog}{\log_2}
\newcommand{\xor}{\oplus}
\newcommand{\minSize}{1.5em}
\newcommand{\gen}[1]{\ensuremath{\langle #1\rangle}}
\newcommand{\noi}{\noindent}

\tikzset{%
	leaf/.style = {draw, fill}, %, minimum size=\minSize},
	empty/.style = {draw},
	wrong/.style = {draw, fill = red},
	internal/.style = {draw, path picture={\draw 
			(path picture bounding box.south east) -- (path picture bounding box.north west) 		(path picture bounding box.south west) -- (path picture bounding box.north east);}}
}

\title{ Baby Jubjub elliptic curve \vspace{-0.2cm} }
\author[1]{Author 1}
\author[2]{Author 2}
\author[1]{Author 3}
\affil[1]{Affiliation 1}
\affil[2]{Affiliation 2}
\date{} %% if you don't need date to appear
\setcounter{Maxaffil}{0}
\renewcommand\Affilfont{\itshape\small}
% \includegraphics[scale=0.3]{iden3.png} 

\begin{document}

\maketitle 
\vspace{1cm}
\tableofcontents

\vspace{0.5cm}

%\newpage

\section{Scope, motivation and background}		% Scope: a section specifying the scope of the standard, highlighting what is being standardized and what is not.
	% Motivation: a section describing at least one concrete application motivating the proposed standard, including an explanation of why the community will benefit from such a standard.
	% Background: a section introducing the problem, including definitions, references to previous work and other background details.	
	Roughly speaking, the Pedersen hash is a secure hash function that maps a sequence of bits to a point in %a large subgroup of 
	an elliptic curve \cite{pedersen-gen}. The big advantage of % using 
	this hash is the fact that it is very efficient to compute inside a circuit, so it suits zk-SNARK very well [REF]. \\
	% Discussion about its efficiency inside circuits:
	% https://github.com/zcash/zcash/issues/2258

	The Pedersen hash has already been defined and used by the ZCash team in Sapling, their latest network upgrade \cite{sapling}. They construct it on Jubjub elliptic curve and using 3-bit lookup tables. In this document, we propose an implementation of the Pedersen hash function using Baby-Jubjub elliptic curve and 4-bit windows, which requires less constraints per bit than using 3-bit windows. \\


\section{Terminology and description}

%Let $p$ be a prime number and $\Fp$ the finite field with $p$ elements. Let $E$ be an elliptic curve defined over $\Fp$ of order $n = h\times r$, where $r$ is a large prime and $h$ is typically a small positive integer called cofactor.\\ 
%% The notation used in this document is the following one.
%
%\begin{table}[htp]
%	\begin{center}
%		\begin{tabular}{c|c|c}
%			{\bf Group} & {\bf Description} & {\bf Order} \\
%			\hline 
%			$\Fp$ & Finite field with $p$ elements & $p$\\
%			$E$ & Elliptic curve group & $n = h\times r$\\
%			$\G \subseteq E$ & Subgroup of points of $E$ of order $r$ & $r$%\\
%		\end{tabular}
%	\end{center}
%\end{table}


Baby-Jubjub elliptic curve is the result of a deterministic algorithm [REF]. 

Appendix A.  Deterministic Generation. 

Specifically, it defines how to generate the parameter
A of the Montgomery curve $y^2 = x^3 + A*x^2 + x$.  

that satisfies SafeCurves criteria.

This procedure is
intended to be as objective as can reasonably be achieved so that
it's clear that no untoward considerations influenced the choice of
curve.  

The input to this process is p, the prime that defines the
underlying field. 

The value (A-2)/4 is used in several of the elliptic curve point
arithmetic formulas.  For simplicity and performance reasons, it is
beneficial to make this constant small, i.e., to choose A so that
(A-2) is a small integer that is divisible by four.





%% Generating a curve where p = 1 mod 4
%\begin{lstlisting}
%def findCurve (prime, curveCofactor, twistCofactor):
%	F = GF(prime)
%	
%	for A in xrange(3, int(1e9)):
%		if (A-2) % 4 != 0:
%			continue
%	
%		try:
%			E = EllipticCurve(F, [0, A, 0, 1, 0])
%		except:
%			continue
%		
%		groupOrder = E.order()
%		twistOrder = 2*(prime+1)-groupOrder
%	
%	if (groupOrder % curveCofactor == 0 and
%		is_prime(groupOrder // curveCofactor) and
%		twistOrder % twistCofactor == 0 and
%		is_prime(twistOrder // twistCofactor)):
%		return A
%	
%def find1Mod4 (prime):
%	assert((prime % 4) == 1)
%	return findCurve(prime, 8, 4)
%\end{lstlisting}

This is the lowest A=168698 of a montgomary curve that statifies the critieria defined in ref7748
This ref: We used the algorithm from appendix A.%
%https://tools.ietf.org/html/rfc7748
We take p as the curve order of bn128curve (REF) to embed them.

Equivalent to (twisted) Edwards.

As
$p=...$
és = 1 mòdul 4.
	\subsection{Baby Jubjub elliptic curve}	Consider the prime number 
$$	p = 21888242871839275222246405745257275088548364
400416034343698204186575808495617 $$
and let $\Fp$ be the finite field with $p$ elements. 

% \subsubsection{Montgomery form}
%{\it{Baby-Jubjub}} is birationally equivalent to the Montgomery elliptic curve defined by  
	$$ E_M : v^2 = u^3 + 168698 u^2 + u. $$
The birational equivalence from $E$ to $E_M$ is the map 
	$$ (x,y) \to (u,v) = \left( \frac{1 + y}{1 - y} , \frac{1 + y}{(1 - y)x} \right) $$
with inverse from $E_M$ to $E$
	$$ (u, v) \to (x, y) = \left(  \frac{u}{v}, \frac{u - 1}{u + 1}   \right). $$
These results are from \cite[Theorem 3.2]{twisted}.
We define $E_M$ as the {\it Baby-Jubjub} Montgomery elliptic curve defined over $\Fp$ given %described
by equation
$$	E: v^2 = u^3 +  168698u^2 + u. $$
The order of $E_M$ is $n = 8\times r$, where 
$$	r = 2736030358979909402780800718157159386076813972
158567259200215660948447373041 $$ 
is a prime number. Denote by $\G$ the subgroup of points of order $r$, that is, %of $E$ 
$$\G = \Set{ P \in E(\Fp) | r P = O  }.$$

% \subsubsection{Edwards form}
$E_M$ is birationally equivalent to the Edwards elliptic curve %[REF] defined by  
$$	E: x^2 + y^2 = 1 +  d x^2 y^2 $$
where
$ d = 9706598848417545097372247223557719406784115219466060233080913168975159366771.$ \\

% \subsubsection{Birational equivalence}

The birational equivalence \cite[Thm. 3.2]{twisted} from $E$ to $E_M$ is the map
% These results are from \cite[Theorem 3.2]{twisted}. 
$$ (x,y) \to (u,v) = \left( \frac{1 + y}{1 - y} , \frac{1 + y}{(1 - y)x} \right) $$
with inverse from $E_M$ to $E$
$$ (u, v) \to (x, y) = \left(  \frac{u}{v}, \frac{u - 1}{u + 1}   \right). $$

		\subsubsection{(TWISTED??) Edwards form}
	We define the twisted Edwards curve {\it Baby-Jubjub} defined over $\Fp$ with 
	$$	p = 21888242871839275222246405745257275088548364
			400416034343698204186575808495617 $$
described by 
	$$	E: 168700 x^2 + y^2 = 1 + 168696 x^2 y^2. $$  

The order of $E$ is $8\times r$, where 
	$$	r = 2736030358979909402780800718157159386076813
			972158567259200215660948447373041. $$ 
The rest of specifications and the satisfiability of SafeCurves criteria  of this curve can be found in \cite{github-barry}. 
		\subsubsection{Montgomery form}
	{\it{Baby-Jubjub}} is birationally equivalent to the Montgomery elliptic curve defined by  
	$$ E_M : v^2 = u^3 + 168698 u^2 + u. $$
The birational equivalence from $E$ to $E_M$ is the map 
	$$ (x,y) \to (u,v) = \left( \frac{1 + y}{1 - y} , \frac{1 + y}{(1 - y)x} \right) $$
with inverse from $E_M$ to $E$
	$$ (u, v) \to (x, y) = \left(  \frac{u}{v}, \frac{u - 1}{u + 1}   \right). $$
These results are from \cite[Theorem 3.2]{twisted}.
	\subsection{Arithmetic in the elliptic curve}
	In this section we define two operations supported in the elliptic curve group: addition of points and multiplication of a point by a scalar (an element of $\Fp$).
	 	\subsubsection{Addition of points} \input{addition}
	 	\subsubsection{Multiplication of a point by a scalar} \input{multiplication}
 
\section{Challenges and security}	%	If relevant, provide a proof of security in the description.
 - It saitsfies the criteria used in such. 
	
\section{Implementation}			%	If relevant, submit an open source prototype implementation (by including a reference to the repository with the code).

% Others (entry?)
Nonce? of a claim?

Types of entries: claims, etc. D'una part en treiem el lloc on ho guardem i a l'altre etc.			
Barry i Jordi codi. Implementació de les operacions.
	
\section {Intellectual Property}	%	We aim to ensure that proposals can be freely implemented. Thus, proposals should disclose the existence of any known patents (awarded or pending) which may restrict free implementation. This may affect the decision process, and a detailed policy is being developed.
	
	\cite{generation-baby}

	\newpage
	\addcontentsline{toc}{section}{References}
	\bibliographystyle{acm}
	\bibliography{lit}

\end{document}