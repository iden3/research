% Bibliografia: patches, unsecure MiMC-7.

\documentclass{article}
\usepackage[colorlinks=true,
			urlcolor=black,
			linkcolor=black,
			citecolor=black
			]{hyperref} 

\usepackage[english]{babel}
\usepackage[utf8]{inputenc}
\usepackage{amsmath, amsthm, amssymb}
\usepackage{enumerate, enumitem}
\usepackage{graphicx}
\usepackage{color, xcolor}
\usepackage{setspace}
\usepackage{hyperref}
\usepackage{authblk}
\usepackage{tikz}
\usetikzlibrary{arrows}
\usetikzlibrary{positioning}
\usepackage{mathtools}
%floor vs. ceil
\DeclarePairedDelimiter{\floor}{\lfloor}{\rfloor} 
\DeclarePairedDelimiter{\ceil}{\lceil}{\rceil}  % If called \ceil*{x} it will add left/right.

%\usepackage{algorithmicx}
\usepackage{algorithm}
\usepackage[noend]{algpseudocode}
\makeatletter
\def\BState{\State\hskip-\ALG@thistlm}
\makeatother
%\usepackage{listings}
%\lstdefinelanguage{Sage}[]{Python}
%{morekeywords={False,sage,True},sensitive=true}
%\lstset{
%  frame=none,
%  showtabs=False,
%  showspaces=False,
%  showstringspaces=False,
%  commentstyle={\ttfamily\color{dgreencolor}},
%  keywordstyle={\ttfamily\color{dbluecolor}\bfseries},
%  stringstyle={\ttfamily\color{dgraycolor}\bfseries},
%  language=Sage,
%  basicstyle={\fontsize{10pt}{10pt}\ttfamily},
%  aboveskip=0.3em,
%  belowskip=0.1em,
%  numbers=left,
%  numberstyle=\footnotesize
%}


\textwidth 16 cm
\textheight 22 cm
\topmargin -1 cm
\oddsidemargin -0 cm

\addtolength{\skip\footins}{1pc plus 5pt} % Foot note space
\setlength\parindent{0pt} % No indent

\newcommand{\N}{\ensuremath{\mathbb{N}}}
\newcommand{\Np}{\ensuremath{\mathbb{N}^{+}}}
\newcommand{\Z}{\ensuremath{\mathbb{Z}}}
\newcommand{\Q}{\ensuremath{\mathbb{Q}}}
\newcommand{\R}{\ensuremath{\mathbb{R}}}
\newcommand{\C}{\ensuremath{\mathbb{C}}}
\newcommand{\Fp}{\ensuremath{\mathbb{F}_p}}
\newcommand{\Fr}{\ensuremath{\mathbb{F}_r}}
\newcommand{\G}{\ensuremath{\mathbb{G}}}
\newcommand{\point}[1]{P_{#1} = (x_{#1}, y_{#1})}
\newcommand{\llog}{\log_2}
\newcommand{\xor}{\oplus}
\newcommand{\minSize}{1.5em}
\newcommand{\gen}[1]{\ensuremath{\langle #1\rangle}}
\newcommand{\noi}{\noindent}

\tikzset{%
	leaf/.style = {draw, fill}, %, minimum size=\minSize},
	empty/.style = {draw},
	wrong/.style = {draw, fill = red},
	internal/.style = {draw, path picture={\draw 
			(path picture bounding box.south east) -- (path picture bounding box.north west) 		(path picture bounding box.south west) -- (path picture bounding box.north east);}}
}

\makeatletter
\renewcommand\AB@affilsepx{, \protect\Affilfont}
\makeatother

\title{ EdDSA For Baby Jubjub Elliptic Curve with MiMC-7 Hash \vspace{-0.2cm} }
\author[1]{Jordi Baylina}
\author[1,2]{Marta Bellés}
\affil[1]{iden3}
\affil[2]{Universitat Pompeu Fabra}
\date{} %% if you don't need date to appear
\setcounter{Maxaffil}{0}
\renewcommand\Affilfont{\itshape\small}
% \includegraphics[scale=0.3]{iden3.png} 

\begin{document}
\begin{spacing}{1.2}
\maketitle 
\vspace{1cm}
\tableofcontents

\vspace{0.5cm}

\newpage

\section{Scope}
This proposal aims to standarize the elliptic curve signature scheme Edwards-curve Digital Signature Algorithm (EdDSA) for Baby Jubjub Edwards elliptic curve using MiMC-7 hash function.

\section{Motivation}
EdDSA is a variant of Schnorr's signature scheme and it provides high performance on a variety of platforms \cite{eddsa}. 

\section{Background}
There are many implementations of EdDSA with Edwards elliptic curves such as Ed25519 or Ed448-Goldilocks and most of them use hash SHA-512. This is the first document specifying a protocol for implementing EdDSA using MiMC-7 and we describe it on the Baby Jubjub Elliptic curve.  \\

The choice of the MiMC-7 hash function makes computations inside circuits very efficient and it has a big potential in zero knowledge protocols such as zk-SNARK.
%	
%\section{Motivation}					% Scope: a section specifying the scope of the standard, highlighting what is being standardized and what is not.
	% Motivation: a section describing at least one concrete application motivating the proposed standard, including an explanation of why the community will benefit from such a standard.
	% Background: a section introducing the problem, including definitions, references to previous work and other background details.	
	Roughly speaking, the Pedersen hash is a secure hash function that maps a sequence of bits to a point in %a large subgroup of 
	an elliptic curve \cite{pedersen-gen}. The big advantage of % using 
	this hash is the fact that it is very efficient to compute inside a circuit, so it suits zk-SNARK very well [REF]. \\
	% Discussion about its efficiency inside circuits:
	% https://github.com/zcash/zcash/issues/2258

	The Pedersen hash has already been defined and used by the ZCash team in Sapling, their latest network upgrade \cite{sapling}. They construct it on Jubjub elliptic curve and using 3-bit lookup tables. In this document, we propose an implementation of the Pedersen hash function using Baby-Jubjub elliptic curve and 4-bit windows, which requires less constraints per bit than using 3-bit windows. \\

%	
%\section{Background}				%	A section introducing the problem, including definitions, references to previous work and other background details.
	
\section{Terminology}	%			%	For consistency across documents, adopt throughout the proposal, terminology and definitions used in the ZKProof proceedings, with pointers to the relevant sections.
	
The table below summarizes the terminology used across the document. Each element is explained in greater detail in the following sections.
	
		\begin {table}[h!]
	\centering
	\begin{tabular}{|l|l|}
		\hline
		{\bf Notation} & {\bf Description}\\
		\hline
		{$p$} & {Prime number.}\\
		{$\Fp$} & {Finite field with $p$ elements.}\\
		{$E$} & {Baby Jubjub elliptic curve (defined over $Fp$) in Edwards form.}\\
		{$E_M$} & {Baby Jubjub elliptic curve (defined over $Fp$) in Montgomery form.}\\
		{$l$} & {Large prime number dividing the order of Baby Jubjub.}\\
		{$\Fl$} & {Finite field with $l$ elements.}\\
		{$\G$} & {Group of $\Fp$-rational points of order $l$.}\\
		{$B$} & {Base point (generator of $\G$) of Baby Jubjub.}\\
		\hline
		{$A = (A_x, A_y)$} & {Public key. $A$ is a point on $E$. }\\
		$k$ & Private key. \\
		$M$ & Message. $M$ is an element of $\Fl$. \\		
		{$(R,S) = ((R_x, R_y), S)$} & Signature on $M$. $R$ is a point on $E$ and $S$ and element of $\Fl$.\\
		\hline		
		{$H$} & {Hash function MiMC-7.}\\
		$r$ & Number of rounds of MiMC-7. \\
		$c_0, c_1, \dots, c_r$ & Constants used in MiMC-7. \\
		\hline
	\end{tabular}
%	\caption{Notation for the EdDSA signature scheme.}
	\label{tab:notation}
	\end{table}
	
	\subsection{Baby-Jubjub} 	Consider the prime number 
$$	p = 21888242871839275222246405745257275088548364
400416034343698204186575808495617 $$
and let $\Fp$ be the finite field with $p$ elements. 

% \subsubsection{Montgomery form}
%{\it{Baby-Jubjub}} is birationally equivalent to the Montgomery elliptic curve defined by  
	$$ E_M : v^2 = u^3 + 168698 u^2 + u. $$
The birational equivalence from $E$ to $E_M$ is the map 
	$$ (x,y) \to (u,v) = \left( \frac{1 + y}{1 - y} , \frac{1 + y}{(1 - y)x} \right) $$
with inverse from $E_M$ to $E$
	$$ (u, v) \to (x, y) = \left(  \frac{u}{v}, \frac{u - 1}{u + 1}   \right). $$
These results are from \cite[Theorem 3.2]{twisted}.
We define $E_M$ as the {\it Baby-Jubjub} Montgomery elliptic curve defined over $\Fp$ given %described
by equation
$$	E: v^2 = u^3 +  168698u^2 + u. $$
The order of $E_M$ is $n = 8\times r$, where 
$$	r = 2736030358979909402780800718157159386076813972
158567259200215660948447373041 $$ 
is a prime number. Denote by $\G$ the subgroup of points of order $r$, that is, %of $E$ 
$$\G = \Set{ P \in E(\Fp) | r P = O  }.$$

% \subsubsection{Edwards form}
$E_M$ is birationally equivalent to the Edwards elliptic curve %[REF] defined by  
$$	E: x^2 + y^2 = 1 +  d x^2 y^2 $$
where
$ d = 9706598848417545097372247223557719406784115219466060233080913168975159366771.$ \\

% \subsubsection{Birational equivalence}

The birational equivalence \cite[Thm. 3.2]{twisted} from $E$ to $E_M$ is the map
% These results are from \cite[Theorem 3.2]{twisted}. 
$$ (x,y) \to (u,v) = \left( \frac{1 + y}{1 - y} , \frac{1 + y}{(1 - y)x} \right) $$
with inverse from $E_M$ to $E$
$$ (u, v) \to (x, y) = \left(  \frac{u}{v}, \frac{u - 1}{u + 1}   \right). $$

	\subsection{MiMC-7}			
%Our elements are of : 

%	, where 
%	$$ r = 	21888242871839275222246405745257275088548364400416034343698204186575808495617 $$
% The hash function used in the Merkle tree is MiMC-7 described in \cite{mimc}.\\
%[REF]. Original paper MiMC hash: \url{https://eprint.iacr.org/2016/492.pdf}.\\
The hash function used in EdDSA is MiMC-7 based in paper \cite{mimc}, which describes the hash using exponent 3. In this specification, we use exponent 7 (hence the name MiMC-7) as 3 and $l-1$ are not coprime and 7 is the optimal choice for exponentiation \cite[Sec. 6]{mimc}.\\

% and goes as follows: %is constructed as follows: 
%
Let $\Fl$ be the finite field with $l$ elements. The block cipher is constructed by iterating a round function $r$ times where each round consists of a key addition with the key $k$, the addition of a round constant $c_i\in \Fr$, and the application of a non-linear function defined as $F(x) :=x^7$ for $x\in \Fl$. The ciphertext is finally produced by adding the key $k$ again to the output of the last round. Hence, the round function is described as $F_i(x) = F(x) \xor k \xor c_i$ where $c_0 = c_r = 0$ and the encryption process is defined as 
	$$ 
		E_k(x) = (F_{r-1} \circ F_{r-2} \circ ... \circ F_0)(x) \xor k.
	$$


%. Instead the constants are fixed once and can be hard-coded into the implementation on either side. 


%The number of rounds is: 
%$$ \text{nRounds} = \ceil*{\frac{\llog r}{\llog 7}} = 91. $$\\

%% Esquema MIMC-7
% !TEX root =/Users/martabellesmunoz/Dropbox/Documents/Especificacions/Merkle trees/Description.tex

\tikzstyle{exp} = [draw, minimum size=2em] % fill=blue!20,
\tikzstyle{init} = [pin edge={to-,thin,black}]
\tikzset{XOR/.style={draw,circle,append after command={
        [shorten >=\pgflinewidth, shorten <=\pgflinewidth,]
        (\tikzlastnode.north) edge (\tikzlastnode.south)
        (\tikzlastnode.east) edge (\tikzlastnode.west)
        }}
}

\begin{center}
\begin{tikzpicture}[node distance=1.5cm, auto,>=latex']

    \node  (in) {$x$};
    \node [XOR, pin={[init]above:$k$}] (xor0) [right of=in, node distance=1cm]  { };
    \node [exp] (e0) [right of=xor0] {$x^7$};
    \node [XOR, pin={[init]above:$k \oplus c_1 $}] (xor1) [right of=e0] { };
    \node [exp] (e1) [right of=xor1] {$x^7$};
    \node [XOR, pin={[init]above:$k \oplus c_{r-1} $}] (xorr-1) [right of=e1, node distance=4cm] { };
    \node [exp] (er-1) [right of=xorr-1] {$x^7$};
    \node [XOR, pin={[init]above:$k \oplus c_r $}] (xor)  [right of=er-1] { };
    \node  (out)  [right of=xor, node distance=1cm] {$y$};

    \path[->] (in) edge node { } (xor0);
    \path[->] (xor0) edge node { } (e0);
    \path[->] (e0) edge node { } (xor1);
    \path[->] (xor1) edge node { } (e1);
    \path[->] (e1) edge[dotted] node { } (xorr-1);
    \path[->] (xorr-1) edge node {} (er-1);
    \path[->] (er-1) edge node { } (xor);
    \path[->] (xor) edge node { } (out);
   
\end{tikzpicture}
\end{center}


As the random  constants $c_i$ do  not  need  to  be  generated  for  every evaluation of MiMC-7, they are hard-coded into the implementation. The generation of these constants and the required number of rounds is described in section \ref{sec-mimc}. 
%
%The hash returns 140 bits, this is why Merkle trees are bounded by 140 levels (it is the length of the whole path of a claim). \\
%
%Encadenar valors (tipu pel hash de H(v1, v2, v3, v4) etc.\\

	\subsection{EdDSA}
	
	The description of this protocol is based in \cite{eddsa}:  
%	
%	Let $k$ be a private key chosen uniformly at random. 
%	
	Let the public key be a point $A = (A_x, A_y)\in E$ of order $l$ and $M$ a message we wish to sign. The signature on $M$ by $A$ consists of a par $(R,S)$ where $R = (R_x, R_y)$ is a point of order $l$ of $E$ and $S\in\Fl\backslash\{0\}$ such that 
			$$ 8SB = 8R + 8H(R,A,M)A.	$$
%	The private key $k$ consists		
		
%		\subsection{Circuit}
	
\section{Challenges and security}
One of the main challenges to create this standard and to see it adopted by the community
is to provide correct, usable, and well-maintained implementations in as many languages as
possible. 
%
Some effort is also required to audit and verify code coming from the community and claiming
to implement EdDSA for Baby Jubjub to prevent the propagation of potentially
insecure implementations. Part of the work in progress of looking batch verification of short signatures. 
%
Lastly, the proposal as it stands uses MiMC-7 as hash function as it works very optimal inside circuits. We believe some work is required to determinate the security MiMC hash functions. 
%
%				%	For motivating the discussions, highlight the main challenges in creating such a standard, as well as any open or unresolved questions. 

% This section is not in the supposed structure of the doc.	
% \section{Description}					The computation of the Pedersen hash has two steps: first, the generation of the base points $P_0, P_1, \dots, P_5$ (we take $k=5$). This only needs to be done only once, as they can be reused to compute hashes of other data {[REF?]}. Secondly, the calculation of expression (\ref{eq-ped}). We describe in terms of circuits how to do such computation and provide an example explaining both steps.  
	% We describe in terms of circuits and with an example, how to do it. 
	% The circuits used to compute this sum are quite similar to the ones used to calculate the multiple of a point of an elliptic curve except that here we only work with the twisted Edwards form of $E$ and we can have many points precalculated, so instead of doubling all the time, we work with look-up tables. 
	


% \section{Security}					%	If relevant, provide a proof of security in the description.
	
\section{Implementation}			% %	If relevant, submit an open source prototype implementation (by including a reference to the repository with the code).

% Others (entry?)
Nonce? of a claim?

Types of entries: claims, etc. D'una part en treiem el lloc on ho guardem i a l'altre etc.
In this section, we specify how each of the main operations in the following EdDSA circuit are computed:
% To compute the following	EdDSA circuit 
\begin{figure}[h]
	\centering
	\includegraphics[scale=0.4]{circuit-eddsa.png}
\end{figure}
% we specify in each how each of the operations are computed. 
%	
%	\subsection{Circuit}
%	\begin{figure}[h]
%	\centering
%		\includegraphics[scale=0.6]{circuit-eddsa.png}
%	\end{figure}

	\subsection{Operations in the elliptic curve}
		\subsubsection{Addition of points} \input{addition}
		\subsubsection{Multiplication of a point of $E$ by a scalar} \input{multiplication}
			\subsection{MiMC-7}
		\label{sec-mimc}
				
				
		The specifications we use in the hash are ({\it we are working in explaining this section in greater detail}):

		\begin{enumerate}
		\item Number of rounds: $ r = \ceil*{\frac{\llog l}{\llog 7}} = 91. $
		
		\item Inputs: \begin{itemize}
					\item Coordinates of the public key: ($A_x, A_y$).
					\item Coordinates of the point $8R$: ($R8_x, R8_y$).
					\item Message $M$. 
				\end{itemize}
		\item Number of inputs: 5.

		\item Generation of constants:  \url{https://github.com/iden3/circomlib/blob/master/src/mimc7.js}.
		\end{enumerate}
			%		exports.getConstants = (seed, nRounds) => {
			%			if (typeof seed === "undefined") seed = SEED;
			%			if (typeof nRounds === "undefined") nRounds = NROUNDS;
			%			const cts = new Array(nRounds);
			%			let c = Web3Utils.keccak256(SEED);
			%			for (let i=1; i<nRounds; i++) {
			%				c = Web3Utils.keccak256(c);
			%				
			%				const n1 = Web3Utils.toBN(c).mod(Web3Utils.toBN(F.q.toString()));
			%				const c2 = Web3Utils.padLeft(Web3Utils.toHex(n1), 64);
			%				cts[i] = bigInt(Web3Utils.toBN(c2).toString());
			%			}
			%			cts[0] = bigInt(0);
			%			return cts;
			%		};
			% https://github.com/iden3/circomlib/blob/master/src/mimc7.js
			% Work in progress
		
	\subsection{Example and test vectors}
	{\it Work in progress.}
	\subsection{Existing implementations}
	EdDSA for Baby Jubjub implemented by Jordi Baylina in circom (zero knowledge circuit compiler):\\ \url{https://github.com/iden3/circomlib/blob/master/circuits/eddsamimc.circom}
\section {Intellectual Property}	%	We aim to ensure that proposals can be freely implemented. Thus, proposals should disclose the existence of any known patents (awarded or pending) which may restrict free implementation. This may affect the decision process, and a detailed policy is being developed.
We will release the final version of this proposal under creative commons, to ensure it is freely available
to everyone.

\addcontentsline{toc}{section}{References}
\bibliographystyle{acm}
\bibliography{lit}
\end{spacing}
\end{document}