
%Our elements are of : 

%	, where 
%	$$ r = 	21888242871839275222246405745257275088548364400416034343698204186575808495617 $$
% The hash function used in the Merkle tree is MiMC-7 described in \cite{mimc}.\\
%[REF]. Original paper MiMC hash: \url{https://eprint.iacr.org/2016/492.pdf}.\\
The hash function used in EdDSA is MiMC-7 based in paper \cite{mimc}, which describes the hash using exponent 3. In this specification, we use exponent 7 (hence the name MiMC-7) as 3 and $l-1$ are not coprime and 7 is the optimal choice for exponentiation \cite[Sec. 6]{mimc}.\\

% and goes as follows: %is constructed as follows: 
%
Let $\Fl$ be the finite field with $l$ elements. The block cipher is constructed by iterating a round function $r$ times where each round consists of a key addition with the key $k$, the addition of a round constant $c_i\in \Fr$, and the application of a non-linear function defined as $F(x) :=x^7$ for $x\in \Fl$. The ciphertext is finally produced by adding the key $k$ again to the output of the last round. Hence, the round function is described as $F_i(x) = F(x) \xor k \xor c_i$ where $c_0 = c_r = 0$ and the encryption process is defined as 
	$$ 
		E_k(x) = (F_{r-1} \circ F_{r-2} \circ ... \circ F_0)(x) \xor k.
	$$


%. Instead the constants are fixed once and can be hard-coded into the implementation on either side. 


%The number of rounds is: 
%$$ \text{nRounds} = \ceil*{\frac{\llog r}{\llog 7}} = 91. $$\\

%% Esquema MIMC-7
% !TEX root =/Users/martabellesmunoz/Dropbox/Documents/Especificacions/Merkle trees/Description.tex

\tikzstyle{exp} = [draw, minimum size=2em] % fill=blue!20,
\tikzstyle{init} = [pin edge={to-,thin,black}]
\tikzset{XOR/.style={draw,circle,append after command={
        [shorten >=\pgflinewidth, shorten <=\pgflinewidth,]
        (\tikzlastnode.north) edge (\tikzlastnode.south)
        (\tikzlastnode.east) edge (\tikzlastnode.west)
        }}
}

\begin{center}
\begin{tikzpicture}[node distance=1.5cm, auto,>=latex']

    \node  (in) {$x$};
    \node [XOR, pin={[init]above:$k$}] (xor0) [right of=in, node distance=1cm]  { };
    \node [exp] (e0) [right of=xor0] {$x^7$};
    \node [XOR, pin={[init]above:$k \oplus c_1 $}] (xor1) [right of=e0] { };
    \node [exp] (e1) [right of=xor1] {$x^7$};
    \node [XOR, pin={[init]above:$k \oplus c_{r-1} $}] (xorr-1) [right of=e1, node distance=4cm] { };
    \node [exp] (er-1) [right of=xorr-1] {$x^7$};
    \node [XOR, pin={[init]above:$k \oplus c_r $}] (xor)  [right of=er-1] { };
    \node  (out)  [right of=xor, node distance=1cm] {$y$};

    \path[->] (in) edge node { } (xor0);
    \path[->] (xor0) edge node { } (e0);
    \path[->] (e0) edge node { } (xor1);
    \path[->] (xor1) edge node { } (e1);
    \path[->] (e1) edge[dotted] node { } (xorr-1);
    \path[->] (xorr-1) edge node {} (er-1);
    \path[->] (er-1) edge node { } (xor);
    \path[->] (xor) edge node { } (out);
   
\end{tikzpicture}
\end{center}


As the random  constants $c_i$ do  not  need  to  be  generated  for  every evaluation of MiMC-7, they are hard-coded into the implementation. The generation of these constants and the required number of rounds is described in section \ref{sec-mimc}. 
%
%The hash returns 140 bits, this is why Merkle trees are bounded by 140 levels (it is the length of the whole path of a claim). \\
%
%Encadenar valors (tipu pel hash de H(v1, v2, v3, v4) etc.\\
