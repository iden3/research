% % !TEX root =/Users/martabellesmunoz/Dropbox/Documents/Berkley-Standards/Structure.tex

\documentclass{article}
\usepackage[colorlinks=true,
			urlcolor=black,
			linkcolor=black,
			citecolor=black
			]{hyperref} 

% !TEX root =/home/marta/Dropbox/Documents/Especificacions/Pedersen/Description.tex

\usepackage[english]{babel}
\usepackage[utf8]{inputenc}
\usepackage{amsmath}
\usepackage{amsthm}
\usepackage{amssymb}
\usepackage{graphicx}
\usepackage{color}
\usepackage{xcolor}
\usepackage{enumerate}
\usepackage{enumitem}
\usepackage{setspace}
\usepackage{hyperref}
%\usepackage{abstract}
%\renewcommand{\abstractname}{}    % clear the title
%\renewcommand{\absnamepos}{empty} % originally center

\textwidth 16 cm
\textheight 22 cm
\topmargin -1 cm
\oddsidemargin -0 cm

\addtolength{\skip\footins}{1pc plus 5pt} % Foot note space
\setlength\parindent{0pt} % No indent

\newcommand{\N}{\ensuremath{\mathbb{N}}}
\newcommand{\Np}{\ensuremath{\mathbb{N}^{+}}}
\newcommand{\Z}{\ensuremath{\mathbb{Z}}}
\newcommand{\Q}{\ensuremath{\mathbb{Q}}}
\newcommand{\R}{\ensuremath{\mathbb{R}}}
\newcommand{\C}{\ensuremath{\mathbb{C}}}
\newcommand{\Fp}{\ensuremath{\mathbb{F}_p}}
\newcommand{\G}{\ensuremath{\mathbb{G}}}
%\newcommand{\maxBits}{{\color{red}{\bf{200 }}}} % 50x4? 200?
\newcommand{\maxBits}{200} % 50x4? 200?
\newcommand{\point}[1]{P_{#1} = (x_{#1}, y_{#1})}

\newcommand{\gen}[1]{\ensuremath{\langle #1\rangle}}

\newcommand{\noi}{\noindent}
\usepackage{authblk}

\title{ Title \vspace{-0.2cm} }
\author[1]{Author 1}
\author[2]{Author 2}
\author[1]{Author 3}
\affil[1]{Affiliation 1}
\affil[2]{Affiliation 2}
\date{} %% if you don't need date to appear
\setcounter{Maxaffil}{0}
\renewcommand\Affilfont{\itshape\small}
% \includegraphics[scale=0.3]{iden3.png} 

\begin{document}

\maketitle 
\vspace{1cm}
\tableofcontents

\vspace{0.5cm}

%\newpage

\section{Scope}						% \input{Scope}
	A section specifying the scope of the standard, highlighting what is being standardized and what is not.
	
\section{Motivation}				% \input{Motivation}
	A section describing at least one concrete application motivating the proposed standard, including an explanation of why the community will benefit from such a standard.
	
\section{Background}				% \input{Background}
	A section introducing the problem, including definitions, references to previous work and other background details.
	
\section{Terminology}				% \input{Terminology}
	For consistency across documents, adopt throughout the proposal, terminology and definitions used in the ZKProof proceedings, with pointers to the relevant sections.
	
\section{Challenges}				% \input{Challanges} 
	For motivating the discussions, highlight the main challenges in creating such a standard, as well as any open or unresolved questions.
	
\section{Security}					% \input{Security}
	If relevant, provide a proof of security in the description.
	
\section{Implementation}			% \input{Implementation}
	If relevant, submit an open source prototype implementation (by including a reference to the repository with the code).
	
\section {Intellectual Property}	% \input{IntellectualProperty}
	We aim to ensure that proposals can be freely implemented. Thus, proposals should disclose the existence of any known patents (awarded or pending) which may restrict free implementation. This may affect the decision process, and a detailed policy is being developed.

%\newpage
%\addcontentsline{toc}{section}{References}
%\bibliographystyle{acm}
%\bibliography{lit}

\end{document}