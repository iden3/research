In \textsc{pedersen hash}, we have depicted the circuit used to compute (equatio \ref{eq-ped}). Each \textsc{multiplication} box returns one term of the sum. 

%that takes, per bit, that many constraints (of a circuit). 

\begin{figure}[h]
	\centering
	\includegraphics[scale=0.4]{Diag/Ped_Hash.png}
	\includegraphics[scale=0.4]{Diag/Ped_Hash_Multiplication.png}
\end{figure}

As the set of generators are fixed, we can precompute its multiples and use 4-bit lookup windows to select the right points. This is done as shown in next circuit \textsc{selector}.

\begin{figure}[h]
	\centering
	\includegraphics[scale=0.5]{Diag/Ped_Hash_Multiplication_selector.png}
\end{figure}

The circuit receives as input a 4-bit chunk. The first three bits are used to select the right multiple of the point and last bit decides the sign of the point. Recall that negation on a point of a twisted Edwards curve corresponds to negation of its first coordinate. 

%
%
%{\bf{Diagram}}	\\\vspace{-0.3cm}				%\input{Mult-diagram}
%
%Let $G$ be one of the generators. Consider a 4-bit window \fbox{$s_0$}\fbox{$s_1$}\fbox{$s_2$}\fbox{$s_3$} . Bit $s_3$ is used to determine the sign and $s_0, s_1, s_2$ the point. \\
%
%
%
%
