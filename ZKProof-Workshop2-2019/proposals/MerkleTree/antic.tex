

\begin{document}


\section{Scope}



\section{Motivation}


\section{Background}


%
%{\it Work in progress.}%
%Merkle trees basically en el que cada nodo que no es una hoja está etiquetado con el hash de la concatenación de las etiquetas o valores (para nodos hoja) de sus nodos hijo.
%that allow the parent node is the hash of its children, and the leaf nodes are hashes of the original data blocks. A hash tree or Merkle tree is a tree in which every leaf node is labelled with the hash of a data block and every non-leaf node is labelled with the cryptographic hash of the labels of its child nodes [REF]. % 

%%We actually deal with Sparse Merkle Trees, but we will refer to them simply as Merkle Trees [REF].\\
%
%This way, only a little piece of data is stored on-chain. And with only this piece of data we can check/validate (with negligible error probability) if a certain piece of data is stored in the tree (or not). 
%De esta forma proporciona un método de verificación segura y eficiente de los contenidos de grandes estructuras de datos. En sus aplicaciones prácticas normalmente el hash del nodo raíz va firmado para asegurar su integridad y que la verificación sea totalmente fiable. La demostración de que un nodo hoja es parte de un árbol hash dado requiere una cantidad de datos proporcional al logaritmo del número de nodos del árbol.
%Hash trees can be used to verify any kind of data stored, handled and transferred in and between computers. They can help ensure that data blocks received from other peers in a peer-to-peer network are received undamaged and unaltered, and even to check that the other peers do not lie and send fake blocks [REF]. proving something is or not in the tree. Merkle trees stuff.

% We store (key, values) from people in what so called Merkle trees. This allows us to ... .
%
%The following pages / This document contains the specifications of the Merkle trees implementation used in iden3 and how to generate and verify the proofs. We also add a section explaining per sobre les possibles concerns i blabla. \\

%	
%\section{Background}				%	A section introducing the problem, including definitions, references to previous work and other background details.
%%	A section introducing the problem, including definitions, references to previous work and other background details.
%
%Problem: storage of big data structures. Store as minimum information as possible. Reference to previous work and other background details? \\
%
%Definition of (sparse) {\bf Merkle tree}: each node consists of a pair (key, value). Our Merkle trees are binary trees % i.e. at most 2 leaves.
%whose nodes consist of pairs {\bf key-value} $(k,v)$. Explain what is a pair key-value. We use them to store claims, which are kept in the leafs and the rest of the nodes furthering up are the hashes of their children. 
%Trees have at most 140 levels, that is, a maximum of $2^140$ claims can be stored in them. This number is determined by the length of the hash function used, we will explain later why).\\



	
\section{Terminology}				% %	For consistency across documents, adopt throughout the proposal, terminology and definitions used in the ZKProof proceedings, with pointers to the relevant sections.

%	For consistency across documents, adopt throughout the proposal, terminology and definitions used in the ZKProof proceedings, with pointers to the relevant sections.



%Two types of proves, when there is an empty node or when there is a different node. 

% and then the nodes further up in the tree 
% Every time a new claim is added to the tree, it is kept in one leaf of the tree and the rest of the nodes 
% furthering up contain the hashes of their children. 
%composed of vertices of $key-value$
% that store the data in their leafs and Nodes further up in the tree are the hashes of their respective children. % All the nodes consist of a pair key-value. 

%
%\begin{itemize}
%	\item Proof of membership
%	\item Proof of non-membership
%	\begin{itemize}
%		\item Case 1: empty node
%		\item Case 2: different node
%	\end{itemize}
%\end{itemize}

\section{Challenges}				%%	For motivating the discussions, highlight the main challenges in creating such a standard, as well as any open or unresolved questions. 
%	For motivating the discussions, highlight the main challenges in creating such a standard, as well as any open or unresolved questions.

{\it Work in progress}.

%Improve such and such.
%
%We included that hash(0,0)=0.
%
%Com que no anem fins avall de tot, p.m.


% This section is not in the supposed structure of the doc.	


%
%Such implementation in GoLang and in JavaScript. \\
%
%The entries are pairs key-value.\\
%
%Example from iden3 documentation.
%	\subsection{Hash}				
%Our elements are of : 

%	, where 
%	$$ r = 	21888242871839275222246405745257275088548364400416034343698204186575808495617 $$
% The hash function used in the Merkle tree is MiMC-7 described in \cite{mimc}.\\
%[REF]. Original paper MiMC hash: \url{https://eprint.iacr.org/2016/492.pdf}.\\
The hash function used in EdDSA is MiMC-7 based in paper \cite{mimc}, which describes the hash using exponent 3. In this specification, we use exponent 7 (hence the name MiMC-7) as 3 and $l-1$ are not coprime and 7 is the optimal choice for exponentiation \cite[Sec. 6]{mimc}.\\

% and goes as follows: %is constructed as follows: 
%
Let $\Fl$ be the finite field with $l$ elements. The block cipher is constructed by iterating a round function $r$ times where each round consists of a key addition with the key $k$, the addition of a round constant $c_i\in \Fr$, and the application of a non-linear function defined as $F(x) :=x^7$ for $x\in \Fl$. The ciphertext is finally produced by adding the key $k$ again to the output of the last round. Hence, the round function is described as $F_i(x) = F(x) \xor k \xor c_i$ where $c_0 = c_r = 0$ and the encryption process is defined as 
	$$ 
		E_k(x) = (F_{r-1} \circ F_{r-2} \circ ... \circ F_0)(x) \xor k.
	$$


%. Instead the constants are fixed once and can be hard-coded into the implementation on either side. 


%The number of rounds is: 
%$$ \text{nRounds} = \ceil*{\frac{\llog r}{\llog 7}} = 91. $$\\

%% Esquema MIMC-7
% !TEX root =/Users/martabellesmunoz/Dropbox/Documents/Especificacions/Merkle trees/Description.tex

\tikzstyle{exp} = [draw, minimum size=2em] % fill=blue!20,
\tikzstyle{init} = [pin edge={to-,thin,black}]
\tikzset{XOR/.style={draw,circle,append after command={
        [shorten >=\pgflinewidth, shorten <=\pgflinewidth,]
        (\tikzlastnode.north) edge (\tikzlastnode.south)
        (\tikzlastnode.east) edge (\tikzlastnode.west)
        }}
}

\begin{center}
\begin{tikzpicture}[node distance=1.5cm, auto,>=latex']

    \node  (in) {$x$};
    \node [XOR, pin={[init]above:$k$}] (xor0) [right of=in, node distance=1cm]  { };
    \node [exp] (e0) [right of=xor0] {$x^7$};
    \node [XOR, pin={[init]above:$k \oplus c_1 $}] (xor1) [right of=e0] { };
    \node [exp] (e1) [right of=xor1] {$x^7$};
    \node [XOR, pin={[init]above:$k \oplus c_{r-1} $}] (xorr-1) [right of=e1, node distance=4cm] { };
    \node [exp] (er-1) [right of=xorr-1] {$x^7$};
    \node [XOR, pin={[init]above:$k \oplus c_r $}] (xor)  [right of=er-1] { };
    \node  (out)  [right of=xor, node distance=1cm] {$y$};

    \path[->] (in) edge node { } (xor0);
    \path[->] (xor0) edge node { } (e0);
    \path[->] (e0) edge node { } (xor1);
    \path[->] (xor1) edge node { } (e1);
    \path[->] (e1) edge[dotted] node { } (xorr-1);
    \path[->] (xorr-1) edge node {} (er-1);
    \path[->] (er-1) edge node { } (xor);
    \path[->] (xor) edge node { } (out);
   
\end{tikzpicture}
\end{center}


As the random  constants $c_i$ do  not  need  to  be  generated  for  every evaluation of MiMC-7, they are hard-coded into the implementation. The generation of these constants and the required number of rounds is described in section \ref{sec-mimc}. 
%
%The hash returns 140 bits, this is why Merkle trees are bounded by 140 levels (it is the length of the whole path of a claim). \\
%
%Encadenar valors (tipu pel hash de H(v1, v2, v3, v4) etc.\\

%	\subsection{Example}			 







%\begin{center}
%\begin{tikzpicture}[auto,node distance=1.5cm] %, sibling distance=10em]
%%  level distance=20mm,
%%  text depth=.1em,
%%  text height=.8em,
%%  level 1/.style={sibling distance=10em},
%%  level 2/.style={sibling distance=20em},
%%  level 3/.style={sibling distance=20em},
%%  level 4/.style={sibling distance=10em}]
% \node [internal] (a){ } %root
%  child { node [internal]  (b) { }
%    		child {node [internal] (d) { }
%			child {node [leaf] (h) { }}
%			child {node [leaf] (i) { }}
%			}
%    		child {node [empty] (e) {asdfasdf }}
%  	}
%   child { node [internal] (c) { }
%    		child {node [empty] (f) { }}
%    		child {node [internal] (g) { }
%			child {node [internal] (j) { }
%				child{node [leaf, fill = green] (l) { } }
%				child{node [leaf] (m) { } }
%			}
%			child {node [empty] (k) { }}	
%		}
%  	};
%\end{tikzpicture}
%\end{center}

	
	
\section {Intellectual Property}	%	We aim to ensure that proposals can be freely implemented. Thus, proposals should disclose the existence of any known patents (awarded or pending) which may restrict free implementation. This may affect the decision process, and a detailed policy is being developed.
We will release the final version of this proposal under creative commons, to ensure it is freely available to everyone.

%\newpage
\addcontentsline{toc}{section}{References}
\bibliographystyle{acm}
\bibliography{lit}

\end{document}